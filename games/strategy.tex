GnuGo
=====

FindOpen (attack)
--------
This recursively explores around a group of pieces, stopping
when an opponent piece or empty square is found. In the latter
case, if this is not a previous KO square, the square is saved in
a vector. If the vector count reaches the liberty count of the 
group (which should imply all open squares found), FindOpen returns 1.

FindNextMove (defense)
------------
   This recurses around a group, finding empty squares and computing
   the effect playing here has on the liberty of the group. The move
   which has the best positive effect is selected, but only saved if
   it exceeds the current best move weight. If the groups liberties
   were only one this move is returned as the chosen move.

   The weight is calculated with the formula:
       (40 + ((newlib-oldlib) - 1) * 50 / (oldlib * oldlib * oldlib));
      
GenMove
-------
FindAttack
   For each opponent piece with liberty less than four
      cnt = 0
      Find the pieces giving opponent liberty (vector tr,tc)
      if opponent group has only one liberty and val<120
         (r,c,val) = (tr[0],tr[0],120)
      else 
	 for u = 0 .. liberties
	    for v = 0 .. liberties
	       if (u != v)
	          if liberties of (tr[u],tc[u]) not zero
                     apply move, find resulting liberties of (tr[v],tc[v])
                        (i.e. liberties of group if opponent plays this)
                     reject move if my liberty one and opponent's non-zero
                     else value = 120 - 20 * opponent liberties
                     if this is a new high value, save (tr[u],tc[u])
                     undo applied move

FindDefense
   loop thru liberties from 1 to 3
      for each of my pieces with that number of liberties
         if FindNextMove returns a higher value
            save the returned move

FindPattern
   This applies a game tree for openings if in the first few moves,
   otherwise it searches the board for matches of a pattern database
   and uses these to choose a move.

Random move, avoiding edges and center, liberty one, fioe and suicide.

==========================================================
Go Explorer
===========

Identify BLOCKS (strongly connected pieces)
Identify CHAINS (blocks connected on diagonals where both adjacent
	nodes are empty or suicide for opponent)
Compute INFLUENCE of each chain
Use influence and chains to determine GROUPS of allied chains
	(essentially blocks of influence)
Determine group strength and size
Compare for each player.

====================================================================
AmiGo
=====
The following are tried in order:

	takeCorner(x, y)
	lookForSave(x, y)
	lookForSaveN(x, y)
	extend(x, y)
	lookForKill(x, y)
	doubleAtari(x, y)
	lookForAttack(x, y)
	threaten(x, y)
	extend2(x, y)
	connectCut(x, y)
	blockCut(x, y)
	cutHim(x, y)
	extendWall(x, y)
	findAttack2(x, y)
	atariAnyway(x, y)
	underCut(x, y)
	dropToEdge(x, y)
	pushWall(x, y)
	reduceHisLiberties(x, y)
	dropToEdge2(x, y)
 
======================================================================
My Ideas

Each block is associated with a strength (number of liberties)
and value (dependent on area enclosed).
We want to reduce the number of chains, I think, but shouldn't
necessarily do the same (or opposite) with blocks.

A block (chain?) with two eyes is (temporarily) invincible. 
If not, control over a region is determined by ignoring blocks
that are adjacent to stronger enemy blocks (unless those blocks
are part of a chain). Something like:

	for each chain
		determine nearby attacking enemy chains
		determine nearby defending chains
		determine fate of chain based on this.

The remaining blocks are then used to determine value based on
increasing territory and strength, and weakening the opponents.


Basically we want to identify blocks, and associate each block
with a influence (dependent on number of liberties), 
Basic shape analysis:

Lines
=====
These are as strong as their number of liberties. Their
value depends on whether they enclose any territory,
and if so, what size. The overall value of a line
could be the sum of the influence on the enclosed
territory. Extending a line can add at most two
liberties; it cannot reduce the number of liberties.
In fact, line extension is the optimal extension for
increasing block liberties (other than making eyes or connecting
to another block).

Diagonals   O
=========  O

Each stone retains all four liberties, but a single enemy
stone can reduce each stone's number of liberties. However,
as long as both common adjacent neighbours are empty, the
two can be connected and used to save each other from capture.

Triangles   O
=========   OO

The smallest non-linear block.
Note that this has seven liberties (one less than a line).

	
